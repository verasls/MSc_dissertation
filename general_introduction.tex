\documentclass[12pt]{article}
\usepackage[utf8]{inputenc}
\usepackage[english]{babel}
\usepackage{fontspec}
\setmainfont{Arial}
\usepackage[margin=3cm]{geometry}
\usepackage{indentfirst}
\setlength\parindent{1.25cm}
\usepackage{setspace}
\linespread{1.5}
\usepackage{apacite}


\begin{document}

\section*{General introduction}

The purposes of this work were, first, to develop regression equations to predict EE and cut-points to classify SA and PAI in severely obese people based on several metrics obtained from accelerometer data; and second, to develop accelerometry-based equations to predict peak ground reaction forces (pGRF) on normal weight to severely obese subjects. In order to attend this goal, this dissertation is structured in four chapters. Chapter I consists of the general introduction, which presented some background information concerning the dissertation main theme and the primary objectives. Chapter II includes a literature review about accelerometers, their calibration process and their use to predict EE and skeletal mechanical loading. Chapter III is composed of two original articles that developed accelerometry-based prediction models to EE and cut-points to classify PAI and also several regression equations to predict pGRF using raw accelerometer data on normal weight to severely obese subjects. Finally, Chapter IV presents the  dissertation general conclusions and future perspectives.

%\pagebreak

%\bibliography{general_introduction}
%\bibliographystyle{fadeup}

\end{document}