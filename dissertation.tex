\documentclass[12pt]{report}
\usepackage[utf8]{inputenc}
\usepackage[english, portuguese]{babel}
\usepackage{fontspec}
\setmainfont{Arial}
\usepackage[margin=3cm]{geometry}
\linespread{1.5}
\usepackage{indentfirst}
\setlength\parindent{1.25cm}
\usepackage{import}
\usepackage[notocbib]{apacite}

\def\blankpage{%
      \clearpage%
      \thispagestyle{empty}%
      \addtocounter{page}{+0}%
      \null%
      \clearpage}

\begin{document}

\renewcommand{\contentsname}{\centerline{\bfseries\Large Table of Contents}}
\tableofcontents

\blankpage

\section*{\hfil Resumo \hfil}
\addcontentsline{toc}{section}{Resumo}
\vspace{1em}

\noindent Acelerómetros são dispositivos que medem as acelerações do corpo e têm sido amplamente adotados para monitorar atividade física. O seu uso mais frequente é determinar o gasto energético (GE) e a intensidade da atividade física (IAF), mas mais recentemente começaram a ser explorados como um meio de estimar a carga mecânica esquelética. No entanto, quase todos os estudos de calibração de acelerómetros foram desenvolvidos para pessoas não obesas, o que prejudica a aplicação de seus resultados nos obesos. Portanto, o objetivo geral deste trabalho foi o de melhorar a precisão da predição do GE e da carga mecânica e da classificação da IAF, especialmente em pacientes obesos. Para isso, o presente trabalho está estruturado em: i) uma revisão de literatura sobre os acelerómetros, seu processo de calibração e sua utilização para predizer o GE e a carga mecânica; ii) um artigo original que objetivou desenvolver equações de regressão para predizer o GE e pontos de corte para classificar a IAF em pessoas obesas severas baseado em várias métricas de aceleração; e iii) um artigo original que objetivou desenvolver equações baseadas em acelerometria para predizer o pico da força de reação do solo (pFRS) em sujeitos de peso normal até obesos severos. Os resultados revelaram que todos os modelos de predição desenvolvidos para o GE e o pFRS e os pontos de corte para classificar a IAF apresentaram uma boa precisão, com resultados similares ou melhores comparados a outros estudos prévios. Em conclusão, os dados de acelerometria permitem predizer precisamente o GE e classificar a IAF em obesos severos e predizer o pFRS em sujeitos de peso normal até obesos severos. Portanto, futuros estudos podem adotar equações de regressão e pontos de corte apropriadamente desenvolvidos para os obesos e também facilmente determinar a intensidade da carga mecânica in condições clínicas com a utilização de acelerómetros.

\vspace{\fill}
\noindent
\textbf{Palavras-chave:} MONITORES DE ATIVIDADE, OBESIDADE, VALIDAÇÃO, CALORIMETRIA INDIRETA, CARGA MECÂNICA

\blankpage

\section*{\hfil Abstract \hfil}
\addcontentsline{toc}{section}{Abstract}
\vspace{1em}

\noindent Accelerometers are small wearable devices that measure body accelerations and have been widely adopted to objectively monitor physical activity. Their most frequent use is to determine energy expenditure (EE) and physical activity intensity (PAI), but more recently they have started to be explored as a way to estimate skeletal mechanical loading. However, almost all accelerometer calibration studies were developed for non-obese people, hindering an accurate prediction of EE and mechanical loading, as well as inducing a misclassification of PAI in obese patients. Therefore, the general aim of this work was to improve the accuracy of EE and mechanical loading prediction and PAI classification, specially in obese patients. For this, the current work is structured in: i) a literature review about accelerometers, their calibration process and their use to predict EE and mechanical loading; ii) an original article which aimed to develop regression equations to predict EE and cut-points to classify PAI in severely obese people based on several accelerometry metrics; and iii) an original article which aimed to develop accelerometry-based equations to predict peak ground reaction forces (pGRF) on normal weight to severely obese subjects. The results revealed that all of our prediction models developed for EE and pGRF prediction and our cut-points for PAI classification presented a good accuracy with similar or better results compared to other previously published studies. In conclusion, accelerometry data allow to accurately predict EE and classify PAI in severely obese people and to predict pGRF in normal weight to severely obese subjects. Therefore, future studies may adopt appropriate regression equations and cut-points developed for obese people and also to easily determine mechanical loading intensity in clinical settings using accelerometers.

\vspace{\fill}
\noindent
\textbf{Keywords:} ACTIVITY MONITOR, OBESITY, VALIDITY, INDIRECT CALORIMETRY, MECHANICAL LOADING
\pagebreak


\section*{\vfill\raggedleft\bfseries 1. General introduction}
\addcontentsline{toc}{section}{General introduction}
\thispagestyle{empty} 
\pagebreak

\section*{General introduction}

There are plenty of evidence supporting the role of physical activity (PA) in health improvement and chronic diseases prevention \cite{Guthold_2018, Warburton_2017, Warburton_2006}. These evidences contributed to the emergence of recommendations about the type, amount and intensity of PA necessary to maintain  or improve health in the general population \cite{WHO_2010}, and also led to the need of accurate methods to assess PA during daily living \cite{Montoye_2000, Plasqui_2013} either subjectively or objectively. Accelerometers are among the most common devices to objectively measure PA \cite{Strath_2013}, but as they only measure the body segment accelerations, their output needs to be translated into more biologically meaningful information by a process called calibration \cite{Welk_2005}.
 
Nowadays, the majority of calibration studies use the accelerometer output to determine some cardio-metabolic parameters such as energy expenditure (EE) and PA intensity (PAI) levels \cite{Migueles_2017}, but among other important uses is the estimation of biomechanical parameters, such a ground reaction force \cite{Neugebauer_2014}. Another important aspect of the accelerometer calibration studies is that their application is only valid for a population similar to those of the utilised sample \cite{Welk_2005}. Obese people present some different characteristics than the non-obese, as a low resting metabolic rate \cite{Byrne_2005}, lower aerobic physical fitness \cite{Souza_2010} and some biomechanical gait alterations \cite{Bode_2019}. As obesity is an increasingly prevalent condition \cite{Guthold_2018}, specific accelerometer calibration studies are needed for this population in order to accurately estimate the PA related parameters to be used to monitor PA, exercise and their effects on health.

Therefore, the purposes of this work were, first, to develop regression equations to predict EE and cut-points to classify sedentary activity and PAI in severely obese people based on several metrics obtained from accelerometer data; and second, to develop accelerometry-based equations to predict peak ground reaction forces (pGRF) on normal weight to severely obese subjects. In order to attend this goal, this dissertation is structured in four chapters. Chapter I consists of the general introduction, which presented some background information concerning the dissertation main theme and the primary objectives. Chapter II includes a literature review about accelerometers, their calibration process and their use to predict EE and skeletal mechanical loading. Chapter III is composed of two original articles that developed accelerometry-based prediction models to EE and cut-points to classify PAI and also several regression equations to predict pGRF using raw accelerometer data on normal weight to severely obese subjects. Finally, Chapter IV presents the  dissertation general conclusions and future perspectives.

\pagebreak
\renewcommand{\bibname}{\centerline{\bfseries\Large References}}
\bibliography{general_introduction}
\bibliographystyle{fadeup}
\pagebreak

\end{document}