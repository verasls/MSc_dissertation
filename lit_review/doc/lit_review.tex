\documentclass[11pt]{article}
\pagenumbering{gobble}


\begin{document}

\section*{Introduction}

Physical activity (PA) has long been established as one of the main contributors to prevent chronic diseases and promote health {\tiny (Kaminsky LA, J Am Hear Assoc. 2014;3(5):e001430; Warburton DER. Curr Opin Cardiol. 2017;32(5):541–56.)}. Evidence shows that lack of PA leads to an increased risk of cardiovascular disease, diabetes, hypertension, osteoporosis, several types of cancer and a higher mortality rate {\tiny (Guthold R. Lancet Glob Heal. 2018;6(10):e1077–86., Lee IM. Lancet. 2012;380(9838):219–29; Shiroma EJ. J Am Heart Assoc. 2014;3(5):7–9.)}. Given the relevant relationship between PA and health, there is an increasing need of accurate and reliable methods of PA assessment on daily life {\tiny (Montoye HJ. Med Sci Sport Exerc. 2000;32(9 Suppl):S439–41.; Plasqui G. Obes Rev. 2013;14(6):451–62; Strath SJ. Circulation. 2013;128(20):2259–79)}. These methods can be either subjective, such as questionnaires, or objective, as direct observation and wearable devices {\tiny (Strath SJ. Circulation. 2013;128(20):2259–79; Troiano RP. Med Sci Sport Exerc. 2005;37(Supplement):S487–9)}.\\

\noindent
The most commonly used wearable devices to assess PA are accelerometers {\tiny (Strath SJ. Circulation. 2013;128(20):2259–79)}, described as equipments that detect the body movement accelerations in one to three orthogonal planes (anteroposterior, mediolateral, and vertical) {\tiny (Chen KY. Med Sci Sport Exerc. 2005;37(Supplement):S490–500)}. The accelerometers output can be either raw acceleration, usually expressed as gravitational acceleration units (\textit{g}), or activity counts, which are processed data derived from the raw acceleration and are based on manufacturer-specific algorithm {\tiny (Chen KY. Med Sci Sport Exerc. 2005;37(Supplement):S490–500; Bassett  Jr. Med Sci Sport Exerc. 2012;44(1 Suppl 1):S32-8; Troiano RP. Br J Sports Med. 2014;48(13):1019–23)}. In both cases, the accelerometer output needs to be translated into more biologically meaningful information through a calibration process {\tiny (Matthews CE. Med Sci Sport Exerc. 2005;37(11 (Suppl)):S512–22.)}. \\

\noindent
Currently, most calibration studies use accelerometer output to relate to energy expenditure (EE) and PA intensity levels {\tiny (Migueles JH. Sport Med. 2017;47(9):1821–45; Mendes MA. Gait Posture. 2018;61:98–110)} but they can be also used to estimate biomechanical parameters, such as ground reaction forces (GRF) {\tiny (Neugebauer JM. PLoS One. 2014;9(6):e99023; Fortune E. J Appl Biomech. 2014;30(5):668–74)}, to evaluate standing balance	{\tiny (Mayagoitia RE. Gait Posture. 2002;16(1):55–9)}, to detect the type of PA being performed {\tiny (Bonomi AG. Med Sci Sport Exerc. 2009;41(9):1770–7; Zhang S. Med Sci Sport Exerc. 2012;44(11):2228–34)}, among others. Apart from the standard measure against which the accelerometer output needs to be compared	, calibration studies must carefully define the sample characteristics, PA protocols and statistical approaches, as these aspects can greatly influence the study internal and external validity {\tiny (Bassett  Jr. Med Sci Sport Exerc. 2012;44(1 Suppl 1):S32-8.; Welk GJ. Med Sci Sport Exerc. 2005;37(Supplement):S501–11)}. This led to the emergence of several different calibration studies in the literature with distinct methods to predict a given outcome variable {\tiny (Mendes MA. Gait Posture. 2018;61:98–110, Matthews. Med Sci Sports Exerc. 2018 Jun 21.)} 	\\

\noindent
This current review aims to describe the current literature regarding the use of accelerometers to measure EE, classify physical activity intensities and estimate GRF, as well as issues about calibration and validation studies of these wearable monitors.

\section*{Accelerometers}

As said before, accelerometers are wearable devices used to measure PA related variables {\tiny (Chen KY. Med Sci Sport Exerc. 2005;37(Supplement):S490–500.)}. The first portable accelerometer was developed in the 1980s {\tiny (Wong TC. IEEE Trans Biomed Eng. 1981;28(6):467–71., Montoye HJ. Med Sci Sport Exerc. 1983;15(5):403–7.)} and with the many technological advances since then, the use of such devices in research is ever growing. An observation of the number of published articles referring to PA or exercise and accelerometer or accelerometry shows that

\end{document}