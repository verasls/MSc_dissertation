\documentclass[12pt]{article}
\usepackage[latin1]{inputenc}
\usepackage[english, portuguese]{babel}
\usepackage{fontspec}
\setmainfont{Arial}
\usepackage[margin=3cm]{geometry}
\linespread{1.5}
\usepackage{sectsty}
\sectionfont{\centering}

\begin{document}

\section*{Resumo}
\vspace{1em}

Acelerómetros são dispositivos vestíveis que medem as acelerações do corpo e tem sido amplamente adotados para monitorar atividade física. O seu uso mais frequente é determinar o gasto energético (GE) e a intensidade da atividade física (IAF), mas mais recentemente eles começaram a ser explorados como um meio de estimar a carga mecânica esquelética. No entanto, quase todos os estudos de calibração de acelerómetros foram desenvolvidos para pessoas não obesas, o que dificulta uma predição precisa do GE e da carga mecânica, e induz uma classificação incorreta da IAF em obesos. Portanto, o objetivo geral deste trabalho foi o de melhorar a precisão da predição do GE e da carga mecânica e da classificação da IAF, especialmente em pacientes obesos. Para isso, o presente trabalho está estruturado em: i) uma revisão de literatura sobre os acelerómetros, seu processo de calibração e sua utilização para predizer o GE e a carga mecânica; ii) um artigo original que objetivou desenvolver equações de regressão para predizer o GE e pontos de corte para classificar a IAF em pessoas obesas severas baseado em várias métricas de aceleração; e iii) um artigo original que objetivou desenvolver equações baseadas em acelerometria para predizer o pico da força de reação do solo (pFRS) em sujeitos de peso normal até obesos severos. Os resultados revelaram que todos os nossos modelos de predição desenvolvidos para predizer o GE e o pFRS e nossos pontos de corte para classificar a IAF apresentaram uma boa precisão, com resultados similares ou melhores comparados a outros estudos prévios. Em conclusão, os dados de acelerometria permitem predizer precisamente o GE e classificar a IAF em obesos severos e predizer o pFRS em sujeitos de peso normal até obesos severos. Esses resultados possibilitam que futuros estudos adotem equações de regressão e pontos de corte apropriadamente desenvolvidos para os obesos ao invés daqueles estabelecidos para pessoas não obesas e também facilmente determinar a intensidade da carga mecânica in condições clínicas com a utilização de acelerómetros.

\vspace{\fill}
\noindent
\textbf{Palavras-chave:} MONITORES DE ATIVIDADE, OBESIDADE, VALIDAÇÃO, CALORIMETRIA INDIRETA, CARGA MECÂNICA

\pagebreak

\section*{Abstract}
\vspace{1em}

Accelerometers are small wearable devices that measure body accelerations and have been widely adopted to objectively monitor physical activity. Their most frequent use is to determine energy expenditure (EE) and physical activity intensity (PAI), but more recently they have started to be explored as a way to estimate skeletal mechanical loading. However, almost all accelerometer calibration studies were developed for non-obese people, hindering an accurate prediction of EE and mechanical loading, as well as inducing a misclassification of PAI in obese patients. Therefore, the general aim of this work was to improve the accuracy of EE and mechanical loading prediction and PAI classification, specially in obese patients. For this, the current work is structured in: i) a literature review about accelerometers, their calibration process and their use to predict EE and mechanical loading; ii) an original article which aimed to develop regression equations to predict EE and cut-points to classify PAI in severely obese people based on several accelerometry metrics; and iii) an original article which aimed to develop accelerometry-based equations to predict peak ground reaction forces (pGRF) on normal weight to severely obese subjects. The results revealed that all of our prediction models developed for EE and pGRF prediction and our cut-points for PAI classification presented a good accuracy with similar or better results compared to other previously published studies. In conclusion, accelerometry data allow to accurately predict EE and classify PAI in severely obese people and to predict pGRF in normal weight to severely obese subjects. These results enable future studies to adopt appropriate regression equations and cut-points developed for obese people rather than those established for non-obese people and also to easily determine mechanical loading intensity in clinical settings using accelerometers.

\vspace{\fill}
\noindent
\textbf{Keywords:} ACTIVITY MONITOR, OBESITY, VALIDITY, INDIRECT CALORIMETRY, MECHANICAL LOADING

\end{document}