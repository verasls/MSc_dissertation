\documentclass[12pt]{article}
\usepackage[utf8]{inputenc}
\usepackage[english]{babel}
\usepackage{fontspec}
\setmainfont{Arial}
\usepackage[margin=3cm]{geometry}
\usepackage{indentfirst}
\setlength\parindent{1.25cm}
\usepackage{setspace}
\linespread{1.5}
\usepackage{apacite}


\begin{document}

\section*{General introduction}

There are plenty of evidence supporting the role of physical activity (PA) in health improvement and chronic diseases prevention \cite{Guthold_2018, Warburton_2017, Warburton_2006}. These evidences contributed to the emergence of recommendations about the type, amount and intensity of PA necessary to maintain  or improve health in the general population \cite{WHO_2010}, and also led to the need of accurate methods to assess PA during daily living \cite{Montoye_2000, Plasqui_2013} either subjectively or objectively. Accelerometers are among the most common devices to objectively measure PA \cite{Strath_2013}, but as they only measure the body segment accelerations, their output needs to be translated into more biologically meaningful information by a process called calibration \cite{Welk_2005}.
 
 Nowadays, the majority of calibration studies use the accelerometer output to determine some cardio-metabolic parameters such as energy expenditure (EE) and PA intensity (PAI) levels \cite{Migueles_2017}, but among other important uses is the estimation of biomechanical parameters, such a ground reaction force \cite{Neugebauer_2014}. Another important aspect of the accelerometer calibration studies is that their application is only valid for a population similar to those of the utilised sample \cite{Welk_2005}. Obese people present some different characteristics than the non-obese, as a low resting metabolic rate \cite{Byrne_2005}, lower aerobic physical fitness \cite{Souza_2010} and some biomechanical gait alterations \cite{Bode_2019}. As obesity is an increasingly prevalent condition \cite{Guthold_2018}, specific accelerometer calibration studies are needed for this population in order to accurately estimate the PA related parameters to be used to monitor PA, exercise and their effects on health.

Therefore, the purposes of this work were, first, to develop regression equations to predict EE and cut-points to classify sedentary activity and PAI in severely obese people based on several metrics obtained from accelerometer data; and second, to develop accelerometry-based equations to predict peak ground reaction forces (pGRF) on normal weight to severely obese subjects. In order to attend this goal, this dissertation is structured in four chapters. Chapter I consists of the general introduction, which presented some background information concerning the dissertation main theme and the primary objectives. Chapter II includes a literature review about accelerometers, their calibration process and their use to predict EE and skeletal mechanical loading. Chapter III is composed of two original articles that developed accelerometry-based prediction models to EE and cut-points to classify PAI and also several regression equations to predict pGRF using raw accelerometer data on normal weight to severely obese subjects. Finally, Chapter IV presents the  dissertation general conclusions and future perspectives.

\pagebreak

\bibliography{general_introduction}
\bibliographystyle{fadeup}

\end{document}