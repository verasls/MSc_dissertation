\documentclass[12pt]{article}
\usepackage[latin1]{inputenc}
\usepackage[english, portuguese]{babel}
\usepackage{fontspec}
\setmainfont{Arial}
\usepackage[margin=3cm]{geometry}
\linespread{1.5}
\usepackage{sectsty}
\sectionfont{\centering}

\begin{document}

\section*{Abstract}
\vspace{1em}

Accelerometers are small wearable devices that measure body accelerations and have been widely adopted to objectively monitor physical activity. Their most frequent use is to determine energy expenditure (EE) and PA intensity (PAI), but more recently they have started to be explored as a way to estimate skeletal mechanical loading. However, almost all accelerometer calibration studies were developed for non-obese people, hindering an accurate prediction of EE and mechanical loading, as well as inducing a misclassification of PAI in obese patients. Therefore, the general aim of this study was to improve the accuracy of EE and mechanical loading prediction and PAI classification, specially in obese patients. For this, the current work is structured in: i) a literature review about accelerometers, their calibration process and their use to estimate EE and mechanical loading; ii) an original article which aimed to develop regression equations to predict EE and cut-points to classify PAI in severely obese people based on several accelerometry metrics; and iii) an original article which aimed to develop accelerometry-based equations to predict peak ground reaction forces (pGRF) on normal weight to severely obese subjects. The results revealed that all of our prediction models developed for EE and pGRF prediction and our cut-points for PAI classification presented a good accuracy with similar or better results compared to other previously published studies. In conclusion, accelerometry data allow to accurately predict EE and classify PAI in severely obese people end to predict pGRF in normal weight to severely obese subjects. These results enable future studies to adopt appropriate regression equations and cut-points developed for class II-III obese people rather than those established for non-obese people and also to easily determine mechanical loading intensity in clinical settings using accelerometers.

\vspace{\fill}
\noindent
\textbf{Keywords:} ACTIVITY MONITOR, OBESITY, VALIDITY, INDIRECT CALORIMETRY, MECHANICAL LOADING

\end{document}