\documentclass[11pt]{article}
\pagenumbering{gobble}

\begin{document}

\section*{Introduction}

Physical activity (PA) has long been established as one of the main contributors to preventing chronic diseases and promoting health {\tiny (Kaminsky LA, J Am Hear Assoc. 2014;3(5):e001430; Warburton DER. Curr Opin Cardiol. 2017;32(5):541–56.)}. Evidence shows that lack of PA leads to increased risk of cardiovascular disease, diabetes, hypertension, osteoporosis, several types of cancer and a higher mortality rate {\tiny (Guthold R. Lancet Glob Heal. 2018;6(10):e1077–86., Lee IM. Lancet. 2012;380(9838):219–29; Shiroma EJ. J Am Heart Assoc. 2014;3(5):7–9.)}. Given the relevant relationship between PA and health, there is a increasing need of accurate and reliable methods of PA assessment on daily life {\tiny (Montoye HJ. Med Sci Sport Exerc. 2000;32(9 Suppl):S439–41.; Plasqui G. Obes Rev. 2013;14(6):451–62; Strath SJ. Circulation. 2013;128(20):2259–79)}. These methods can be either subjective, such as questionnaires, or objective, as direct observation and activity monitors {\tiny (Strath SJ. Circulation. 2013;128(20):2259–79; Troiano RP. Med Sci Sport Exerc. 2005;37(Supplement):S487–9)}.

\end{document}