\documentclass[11pt]{article}
\pagenumbering{gobble}
\setlength\parindent{0pt}

\usepackage{graphicx}
\usepackage{apacite}

\begin{document}

\section*{Introduction}

Physical activity (PA) has long been established as one of the main contributors to prevent chronic diseases and promote health \cite{Kaminsky_2014, Warburton_2017}. Evidence shows that lack of PA leads to an increased risk of cardiovascular disease, diabetes, hypertension, osteoporosis, several types of cancer and a higher mortality risk \cite{Guthold_2018, Lee_2012, Shiroma_2014}. Given the relevant relationship between PA and health, there is an increasing need of accurate and reliable methods of PA assessment on daily life \cite{Montoye_2000, Plasqui_2013, Strath_2013}. These methods can be either subjective, such as questionnaires, or objective, as direct observation and wearable devices \cite{Strath_2013, Troiano_2005}. \\

The most commonly used wearable devices to assess PA are accelerometers \cite{Strath_2013}. These are equipments that detect the body movement accelerations in one to three orthogonal planes (anteroposterior, mediolateral, and vertical) \cite{Chen_2005}. The accelerometers output can be either raw acceleration, usually expressed as gravitational acceleration units (\textit{g}), or activity counts (AC), which are processed data derived from the raw acceleration and are based on manufacturer-specific algorithm \cite{Chen_2005, Basset_2012, Troiano_2014}. In both cases, the accelerometer output needs to be translated into more biologically meaningful information through a calibration process \cite{Matthews_2005}. \\

Currently, most calibration studies use the accelerometer output to determine energy expenditure (EE) and PA intensity levels \cite{Migueles_2017, Mendes_2018} but they can be also used to estimate biomechanical parameters, such as ground reaction forces (GRF) \cite{Neugebauer_2014, Fortune_2014}, to evaluate standing balance \cite{Mayagoitia_2002}, to detect the type of PA being performed \cite{Bonomi_2009, Zhang_2012}, among others. Apart from the standard measure against which the accelerometer output needs to be compared, calibration studies must carefully define the sample characteristics, PA protocols and statistical approaches, as these aspects can greatly influence the study internal and external validity \cite{Basset_2012, Welk_2005}. This led to the emergence of several different calibration studies in the literature with distinct methods to predict a given outcome variable \cite{Mendes_2018, Matthews_2018}. \\

This current review aims to describe the literature regarding the use of accelerometers to measure EE, classify physical activity intensities and estimate GRF, as well as issues about calibration and validation studies of these wearable monitors. 

\section*{Accelerometers}

As said previously, accelerometers are wearable devices used to measure PA related variables \cite{Chen_2005}. The first portable accelerometer was developed in the 1980s \cite{Wong_1981, Montoye_1983} and with the many technological advances since then, the use of such devices in research is ever growing, with a major increase in the number of published articles mentioning PA or exercise and accelerometers since the early 2000s (Figure \ref{art_year}).

\begin{figure}[h!]
	\includegraphics[width=\linewidth]{../figs/fig1.tiff}
	\caption{Articles published by year with search terms ``exercise or physical activity'' and ``accelerometer or accelerometry'', Scopus.com, accessed 15 April 2019.}
	\label{art_year}
\end{figure}

From a technical standpoint, the working principle of most accelerometers is based on a sensing element, the seismic mass, and a piezoelectric element. When this system suffers an acceleration, the seismic mass causes a deformation in the piezoelectrical element, generating an output voltage signal proportional to the applied acceleration \cite{Chen_2005, Yang_2010}. The rate by which this data is acquired is determined by the sampling frequency of the device, making it responsive not only to the acceleration intensity, but to its frequency \cite{Mathie_2004}. The signal is then filtered and processed before being digitally stored by the equipment \cite{Chen_2005}. \\

As an objective method to measure PA, accelerometers offer some advantages over subjective methods, such as questionnaires and PA diaries. They are capable of long term continuous data collection with low subject burden \cite{Chen_2012, Strath_2013}, are more accurate than questionnaires to measure PA and sedentary behaviour \cite{Celis-Morales_2012, Matthews_2018} and can measure the full spectrum of daily activities, providing more detailed intensity, frequency and duration data \cite{Matthews_2018, Strath_2013}. \\

On the other hand, accelerometers also present some weaknesses. They cannot reliably account for some activities, such as cycling, climbing stairs, weight-lifting and upper-body activities when worn at hip or lower back \cite{Strath_2013}. Also, different manufacturers have distinct proprietary algorithms to compute raw data into AC, hindering the comparison of results between devices \cite{Plasqui_2013}. In addition, there is still no consensus on the best positioning place for accelerometers on the body, and how to process the data \cite{Troiano_2014}. \\

Some decisions need to be made in order to conduct a research utilizing accelerometers, such as the device sampling frequency, body placement, and which accelerometer output data to use (either AC or raw acceleration). To assure that all human movements are correctly captured by the monitor, the sampling frequency must be at least twice the highest movement frequency, accordingly to the Nyquist criterion \cite{Chen_2012}. The first accelerometry studies utilized a sampling frequency of 30 Hz, which was the acquisition limit of most equipments at that time, but nowadays the recommendation ranges from 90 to 100 Hz \cite{Migueles_2017}. \\

As to the body placement in which to wear the accelerometers, the waist is the most common as it is the closest to the body center of mass \cite{Chen_2005, Mendes_2018}. Other placements are the lower back \cite{Brandes_2012}, wrist \cite{Hildebrand_2017}, ankle \cite{Fortune_2014} and thigh \cite{Montoye_2016a}. Among these body placements, the wrist has been gaining popularity on the past few years, most due to its increase in prediction accuracy in some recent studies \cite{Phillips_2013, Hildebrand_2017} and its higher compliance compared with waist placement. These facts made the United States National Health and Nutrition Examination Survey (NHANES) change from waist to wrist placement in the 2011-2012 and 2013-2014 survey cycles \cite{Troiano_2014}. \\

Early studies on accelerometry utilized the AC since this was the only available output variable at that time. Despite presenting moderate to high correlations with measured EE \cite{Nichols_1999, Freedson_1998} and GRF \cite{Janz_2003}, AC calculation depends on proprietary algorithms that vary among manufacturers, causing different accelerometers to produce different count values even when measuring the same accelerations \cite{Chen_2012, Plasqui_2013}. \\

Recent technological advances have enabled to collect and store raw acceleration data at high frequencies, eliminating the need to summarize them into AC \cite{Bakrania_2016}. The use of raw acceleration entails some advantages, as the ability to extract time-domain and frequency-domain features from the data, allowing to apply more advanced statistical and computational techniques in the calibration process \cite{John_2013}. It can also enhance comparability among accelerometers from different manufacturers \cite{Mendes_2018, Rowlands_2016}. \\

With these modern technologies and the recent endorsement to the use of raw acceleration \cite{Freedson_2012}, several metrics based on raw acceleration have been developed, as the euclidean norm minus one (ENMO) \cite{vanHees_2013}, the mean amplitude deviation (MAD) \cite{Vaha-Ypya_2015} and the activity index (AI) \cite{Bai_2016}. The use of these new metrics also complies with the recommendations for more transparency \cite{Intille_2012}, since they are nonproprietary metrics, with known properties and can be computed using open-source software.

\section*{Calibration of accelerometers}

As already discussed, the accelerometers output can be either AC, a dimensionless unit, or raw acceleration, usually expressed as gravitational acceleration units (\textit{g}). Both outputs can indicate overall movement, but they need to be converted into more biologically meaningful units in a process designated as calibration \cite{Welk_2005}. \\

 This calibration process can be either value or unit calibration. Value calibration examines the accelerometers validity, comparing its measure with a gold standard criterion measure \cite{Basset_2012}. Unit calibration analyses the accelerometer reliability by measuring differences in the output among distinct units of the same device and aims to reduce interinstrument variability \cite{Welk_2005, Basset_2012}. The remaining of this section will address value calibration. \\

To execute a calibration research scientists need to simultaneously collect accelerometer and criterion data on multiple subjects performing different activities. These data will then be used to convert the accelerometer signal into EE estimates, time spent in different PA intensity categories, GRF, activity types, or other physiological or biomechanical variable \cite{Basset_2012}. Regarding the criterion measure, several methods can be used to validate EE predictions, such as direct observation, doubly labeled water, room calorimetry and indirect calorimetry, with the latest being the most used \cite{Basset_2012, Mendes_2018}, and studies validating GRF prediction use force plates as criterion measure \cite{Neugebauer_2018, Neugebauer_2014, Fortune_2014}. \\

Initial calibration research utilized controlled laboratory studies to evaluate accelerometer and criterion data agreement, while in more recent studies the use of free-living activities is becoming more common \cite{Welk_2005, Matthews_2005}. Typical laboratory-based calibration studies, such as those performed by Freedson et al. \citeyear{Freedson_1998} and Nichols et al. \citeyear{Nichols_1999}, used progressively increasing speeds on a treadmill, ranging from slow walking to running, and as result they usually displayed strong associations between AC and measured EE using linear regression. \\

Because locomotor activities are not the only tasks performed on daily living, laboratory research fails to provide a true evaluation of how well accelerometers perform under real-world conditions \cite{Welk_2005}. Thus, many studies have also assessed accelerometers validity using free-living activities in their test protocol. These studies usually include in their procedures a variety of sedentary activities (e.g., lying down, sitting, reading), lifestyle activities (e.g., sweeping, laundering, stair climbing), and exercise activities (e.g., cycling, jumping jacks, squatting) along with locomotor activities \cite{Montoye_2015, Montoye_2016b}. \\

Despite the benefit of including activities that represent more accurately real-world conditions, studies that utilized free-living activities have to consider that the relationship between the accelerometer output and the criterion measure for these activities is so different than for locomotor activities that a single regression equation may not fully characterize the data \cite{Welk_2005}. To address this issue, Crouter et al. \citeyear{Crouter_2006} developed a two-regression method that discriminates locomotor from lifestyle activities based on the AC coefficient of variation. Crouter's model exhibited an improved accuracy compared to the methods available at that time and, since then, novel approaches based on pattern recognition have been developed, some of them even more accurate than the two-regression model \cite{Farrahi_2019, Basset_2012}. \\

Some attention must be paid to certain methodological aspects of accelerometer calibration studies. First, the study sample must be representative of the population in terms of age, weight or body mass index (BMI), and behavioral patterns \cite{Welk_2005}. Thus, a few calibration studies have been done for specific populations such as children \cite{Phillips_2013, McMurray_2016}, adults \cite{Freedson_1998, Hibbing_2018}, obese people \cite{Aadland_2012}, and elderly \cite{Evenson_2015}. \\

Second, since the accelerometer positioning on the body is one of the factors influencing its output, distinct calibration needs to be done for each positioning \cite{Welk_2005}. Furthermore, as accelerometers output can also vary among different units of the same device, a calibration research should employ multiple monitors to allow this variability and avoid bias \cite{Welk_2005}. Finally, a wide range of PA, representing those usually performed by the target population, should be performed during calibration procedures \cite{Welk_2005, Basset_2012}. Additionally, these activities should also include lying, sitting and standing tasks and cover the entire range of intensities, from sedentary to vigorous PA \cite{Basset_2012}. \\

Another key aspect of calibration studies is the statistical approach utilized. To translate the accelerometer output into an outcome variable (e.g., estimates of EE or GRF), a usual method is to develop a regression model, mostly a linear regression \cite{Montoye_2017}. But, as the vast majority of calibration studies use multiple data points for each individual, they violate the linear regression assumption of independence \cite{Welk_2005, Field_2012}. One way to resolve this issue is to apply the linear mixed model (LMM) approach, which allow the repeated data to be modeled in the analysis \cite{Field_2012}. Also, mixed models have the benefit of permitting quadratic and cubic polynomial simulations to be tested \cite{Field_2012}. \\

The advantages provided by the application of mixed models have made a significant contribution to the calibration studies, but nowadays more advanced methods are gaining popularity, such as machine-learning techniques \cite{Montoye_2017, Troiano_2014}. These techniques have the ability to model not just the accelerometer output, but to use statistical summaries of the data in time and frequency domains to describe more thoroughly the acceleration pattern \cite{Staudenmayer_2015, Farrahi_2019}. Thus, machine-learning techniques have the potential to increase accelerometers prediction accuracy, mainly for sedentary and non-locomotor activities, where regression-based models show not to work well \cite{Montoye_2017}. However, if both machine-learning and regression models can achieve similar accuracy, the regression models should be preferred, as they are simpler to apply and interpret \cite{Montoye_2017}. \\

Finally, the accelerometer calibration results must have its performance evaluated \cite{Basset_2012}. These evaluations should ideally be conducted using another sample from the target population, however, as the process to recruit, assess and analyze data from another sample could be costly, a common approach is to use a split-sample cross-validation method \cite{Staudenmayer_2012}. One strategy would be to divide the study sample in two parts, one for calibration and another for cross-validation, but this could be a problem particularly when dealing with small sample sizes \cite{Staudenmayer_2012}. A procedure to overcome this situation is the leave-one-out cross-validation (LOOCV), in which one participant's data is separated in a testing dataset, with the remaining participants in the training dataset \cite{Staudenmayer_2012}. This procedure is repeated until each participant is used in the testing dataset.

\section*{Energy expenditure prediction and physical activity intensity classification}

\pagebreak

\bibliography{bibliography}
\bibliographystyle{apacite}

\end{document}